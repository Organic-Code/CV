% %!TEX program = lualatex

\PassOptionsToPackage{letterpaper}{geometry}
\documentclass[10pt,sans]{moderncv}
\usepackage[utf8]{inputenc}
\usepackage[top=0.85cm, bottom=0.85cm, left=0.85cm, right=0.85cm]{geometry}
\usepackage{graphicx}
\usepackage[firstyear=2009,lastyear=2019]{moderntimeline}
\usepackage{datenumber,fp}
\usepackage{fontawesome}

%----------------------------------------------------------------------------------------
%   SETTINGS
%----------------------------------------------------------------------------------------

%\makeatletter % changes the catcode of @ to 11
%modify here packages internal commands
%\makeatother % changes the catcode of @ back to 12

\makeatletter

%-------------------------------------------
%   Add preinfo to header
%-------------------------------------------

\newcommand*{\preinfo}[1]{\def\@preinfo{#1}}

% optional maketitle width to force a certain width (if set to 0pt, the width is calculated automatically)
\newlength{\makecvtitlenamewidth}
\setlength{\makecvtitlenamewidth}{0pt}% dummy value
\renewcommand*{\makecvtitle}{%
	{\hypersetup{urlcolor=color2}
	% recompute lengths (in case we are switching from letter to resume, or vice versa)
	\recomputecvlengths%
	% optional detailed information (pre-rendering)
	\def\phonesdetails{}%
	\collectionloop{phones}{% the key holds the phone type (=symbol command prefix), the item holds the number
		\protected@edef\phonesdetails{\phonesdetails\protect\makenewline\csname\collectionloopkey phonesymbol\endcsname\collectionloopitem}}%
	\def\socialsdetails{}%
	\collectionloop{socials}{% the key holds the social type (=symbol command prefix), the item holds the link
		\protected@edef\socialsdetails{\socialsdetails\protect\makenewline\csname\collectionloopkey socialsymbol\endcsname\collectionloopitem}}%
	\newbox{\makecvtitledetailsbox}%
	\savebox{\makecvtitledetailsbox}{%
		\addressfont\color{color2}%
		\begin{tabular}[b]{@{}r@{}}%
			\ifthenelse{\isundefined{\@preinfo}}{}{\makenewline{\@preinfo}}%
			\ifthenelse{\isundefined{\@addressstreet}}{}{\makenewline\addresssymbol\@addressstreet%
			\ifthenelse{\equal{\@addresscity}{}}{}{\makenewline\@addresscity}% if \addresstreet is defined, \addresscity and addresscountry will always be defined but could be empty
			\ifthenelse{\equal{\@addresscountry}{}}{}{\makenewline\@addresscountry}}%
			\phonesdetails% needs to be pre-rendered as loops and tabulars seem to conflict
			\ifthenelse{\isundefined{\@email}}{}{\makenewline\emailsymbol\emaillink{\@email}}%
			\ifthenelse{\isundefined{\@homepage}}{}{\makenewline\homepagesymbol\httplink{\@homepage}}%
			\socialsdetails% needs to be pre-rendered as loops and tabulars seem to conflict
			\ifthenelse{\isundefined{\@extrainfo}}{}{\makenewline\@extrainfo}%
		\end{tabular}
	}%
	% optional photo (pre-rendering)
	\newbox{\makecvtitlepicturebox}%
	\savebox{\makecvtitlepicturebox}{%
		\ifthenelse{\isundefined{\@photo}}%
			{}%
			{%
				\hspace*{\separatorcolumnwidth}%
				\color{color1}%
				\setlength{\fboxrule}{\@photoframewidth}%
				\ifdim\@photoframewidth=0pt%
					\setlength{\fboxsep}{0pt}\fi%
				\framebox{\includegraphics[width=\@photowidth]{\@photo}}%
			}%
	}%
	% name and title
	\newlength{\makecvtitledetailswidth}\settowidth{\makecvtitledetailswidth}{\usebox{\makecvtitledetailsbox}}%
	\newlength{\makecvtitlepicturewidth}\settowidth{\makecvtitlepicturewidth}{\usebox{\makecvtitlepicturebox}}%
	\ifthenelse{\lengthtest{\makecvtitlenamewidth=0pt}}% check for dummy value (equivalent to \ifdim\makecvtitlenamewidth=0pt)
		{\setlength{\makecvtitlenamewidth}{\textwidth-\makecvtitledetailswidth-\makecvtitlepicturewidth}}%
		{}%
	\begin{minipage}[b]{\makecvtitlenamewidth - 10mm}%
		\raggedright
		\titlestyle{\@title}\\
		\hfill\\
		\namestyle{\@firstname\ \@lastname}
	\end{minipage}%
	\hfill%
	% optional detailed information (rendering)
	\llap{\usebox{\makecvtitledetailsbox}}% \llap is used to suppress the width of the box, allowing overlap if the value of makecvtitlenamewidth is forced
	% optional photo (rendering)
	\usebox{\makecvtitlepicturebox}\\[2.5em]%
	% optional quote
	\ifthenelse{\isundefined{\@quote}}%
		{}%
		{{\centering\begin{minipage}{\quotewidth}\vspace*{\quoteDeletedSpace}\centering\quotestyle{\@quote}\end{minipage}\\[2.5em]}}%
	\par% to avoid weird spacing bug at the first section if no blank line is left after \makecvtitle
	}
}

%-------------------------------------------
%   Change tllabelcventry label position
%-------------------------------------------

\renewcommand{\tllabelcventry}[9][color1]{%
\tl@formatendyear{#3}
\tl@formatstartyear{#2}
 \cventry{\tikz[baseline=0pt]{
     \fill [\tl@runningcolor] (0,0)
        rectangle (\hintscolumnwidth,\tl@runningwidth);
     \useasboundingbox (0,-1.5ex)
        rectangle (\hintscolumnwidth,1ex);
     \fill [#1] (0,0)
        ++(\tl@startfraction*\hintscolumnwidth,0pt)
        node [tl@singleyear,anchor=north] {#4}
        rectangle (\tl@endfraction*\hintscolumnwidth,\tl@width-1pt) ;
    \ifissince
       \newdimen\fullcolorwidth
       \pgfmathsetlength\fullcolorwidth{\tl@startfraction*(1+(1-\tl@startfraction)*\tl@nsfrac)*\hintscolumnwidth}
       \shade [left color=#1,right color=#1]
(\tl@startfraction*\hintscolumnwidth,0)
           rectangle (\fullcolorwidth,\tl@width);
       \shade [left color=#1] (\fullcolorwidth,0)
           rectangle (\tl@endfraction*\hintscolumnwidth,\tl@width);
     \else
        \fill [#1] (\tl@startfraction*\hintscolumnwidth,0)
            rectangle (\tl@endfraction*\hintscolumnwidth,\tl@width);
     \fi
     }
}
{#5}{#6}{#7}{#8}{#9}%
}

\makeatother

%-------------------------------------------
%   Theme related colors
%-------------------------------------------

% color0      main default color, normally left to black
% color1      primary scheme color
% color2      secondary scheme color

%-------------------------------------------
%   Theme font and style redefinition
%-------------------------------------------

\AtBeginDocument{%

% fonts
\renewcommand*{\titlefont}{\fontsize{34}{14}\mdseries\upshape}
\renewcommand*{\namefont}{\fontsize{19}{14}\mdseries\upshape}
%\renewcommand*{\addressfont}{\normalsize\mdseries\upshape}
%\renewcommand*{\quotefont}{\large\slshape}
%\renewcommand*{\sectionfont}{\Large\bfseries\upshape}
%\renewcommand*{\subsectionfont}{\large\upshape\fontseries{sb}\selectfont}
%\renewcommand*{\hintfont}{\bfseries}

% styles
\renewcommand*{\titlestyle}[1]{{\titlefont\textcolor{color1}{#1}}}
\renewcommand*{\namestyle}[1]{{\namefont\textcolor{color2!85}{#1}}}
%\renewcommand*{\addressstyle}[1]{{\addressfont\textcolor{color2}{#1}}}
%\renewcommand*{\quotestyle}[1]{{\quotefont\textcolor{color1}{#1}}}
%\renewcommand*{\sectionstyle}[1]{{\sectionfont\textcolor{color1}{#1}}}
%\renewcommand*{\subsectionstyle}[1]{{\subsectionfont\textcolor{color1}{#1}}}
%\renewcommand*{\hintstyle}[1]{{\hintfont\textcolor{color0}{#1}}}

}

%-------------------------------------------
%   hyperref colors
%-------------------------------------------

\AtBeginDocument{%

\hypersetup{
	colorlinks=true,
	%linkcolor=red,
	%anchorcolor=black,
	citecolor=green,
	%filecolor=cyan,
	%menucolor=red,
	urlcolor=color1,
	linktoc=all
}

}

%-------------------------------------------
%   Symbols from FontAwesome
%-------------------------------------------

\AtBeginDocument{%

\renewcommand*{\addresssymbol}       {\faMapMarker\ }
\renewcommand*{\mobilephonesymbol}   {\faMobile\ }
\renewcommand*{\fixedphonesymbol}    {\foPhone\ }
\renewcommand*{\faxphonesymbol}      {\faFax\ }
\renewcommand*{\emailsymbol}         {\faEnvelopeO\ }
\renewcommand*{\homepagesymbol}      {\faGlobe\ }
\renewcommand*{\linkedinsocialsymbol}{\faLinkedin\ }
\renewcommand*{\twittersocialsymbol} {\faTwitter\ }
\renewcommand*{\githubsocialsymbol}  {\faGithub\ }

}

%-------------------------------------------
%   Commands definitions
%-------------------------------------------

% Used to indicate the level in the languages section
\newcommand{\cvskill}[2]{%
\textcolor{color2}{\textbf{#1}}
\foreach \x in {1,...,5}{%
	\space{\ifnumgreater{\x}{#2}{\color{color1!30}}{\color{color1}}\faCircle}}%
}

\newcommand{\cvitemcom}[2]{%
	\cvitemwithcomment{}{\listitemsymbol #1}{#2}%
}

% Colored href
\newcommand{\Chref}[3][blue]{\href{#2}{\color{#1}{#3}}}%

% Space to delete between the sections
\newcommand{\deletedSpace}{0mm}

% Space to delete before the quote
\newcommand{\quoteDeletedSpace}{0mm}

%-------------------------------------------
%   Age computation
%-------------------------------------------

\newcounter{birthday}
\newcounter{today}
\setmydatenumber{birthday}{1996}{10}{07}
\setmydatenumber{today}{\the\year}{\the\month}{\the\day}
\FPsub\result{\thetoday}{\thebirthday}
\FPdiv\myage{\result}{365.2425}
\FPtrunc\myage{\myage}{0}

%-------------------------------------------
%   ModernCV theme
%-------------------------------------------

\moderncvtheme[blue]{banking}
%\moderncvtheme[blue]{classic}
%\renewcommand{\familydefault}{\sfdefault}
\nopagenumbers{}

% Space to delete between the sections
%\renewcommand{\deletedSpace}{0mm}

% Space to delete before the quote
%\renewcommand{\quoteDeletedSpace}{0mm}

%----------------------------------------------------------------------------------------
%   PERSONAL DATA
%----------------------------------------------------------------------------------------

\preinfo{%
	%\faUser\ \myage{} ans, célibataire
}
\firstname{Lucas}
\familyname{Lazare}
\title{Élève ingénieur en dernière année d'informatique}
\address{Saguenay, Québec, Canada}{}
\phone[mobile]{+33~6~38~68~21~51}
%\phone[fixed]{}
%\phone[fax]{}
\email{lazarelucas@yahoo.fr}
\social[github]{organic-code}
\quote{En recherche de stage de 6 mois commençant à partir d'août/septembre 2019.}


%----------------------------------------------------------------------------------------
%   DOCUMENT BODY
%----------------------------------------------------------------------------------------

\begin{document}
	\maketitle

	%-------------------------------------------

	\section{Cursus}
		\tlcventry{2018/9}{0}
            {Maîtrise en informatique, Bac+5}
			{Université du Québec à Chicoutimi}
			{Saguenay, Québec, Canada}
			{}
			{\textit{Actuellement en dernière année, double diplomation avec l'UTBM}}
		\tlcventry{2016/9}{0}
            {Diplôme d'ingénieur en informatique, spécialité imagerie, Bac+5}
			{Université de Technologie de Belfort-Montbéliard}
			{Belfort, France}
            {}
			{\textit{Actuellement en dernière année, double diplomation avec l'UQAC}}
		\tlcventry{2014/9}{2016/8}
            {DEUTEC, Bac+2}
			{Université de Technologie de Belfort-Montbéliard}
			{Sevenans, France}
			{Cycle préparatoire intégré}
            {}
		\tlcventry{2009/10}{2011/06}
			{3e année de collège en famille d'accueil en milieu anglophone}
			{Collège Ninu Cremona}
			{Victoria, Malte}
			{}
			{}

	%-------------------------------------------

	%-------------------------------------------

	\vspace*{\deletedSpace}
	\section{Expériences Professionnelles}
		\tllabelcventry{2017/9}{2018/2}{09/2017 - 02/2018}
			{Stage}
			{Laboratoire ICube, 6 mois}
			{Strasbourg, France}
			{}
			{}
			\vspace{-5pt}
			Modélisation 3D de molécules à l'aide de clichés 2D obtenus par cryo-microscopie \\
			électronique, en utilisant la transformée de Fourier (optimisation par recuit simulé, C++)
			\vspace{5pt}
			
		\tllabelcventry{2016/7}{2016/8}{08/2016}
			{Animateur}
			{ASLV, un mois}
			{Mesnil-Saint-Père, France}
			{Colonie de vacances pour adultes en difficulté physique et mentale}
			{}
		\tllabelcventry{2015/7}{2015/8}{08/2015}
			{Animateur}
			{ASLV, un mois}
			{Livry, France}
			{Colonie de vacances pour adultes en difficulté physique et mentale}
			{}
		\tllabelcventry{2015/1}{2015/2}{01/2015}
		    {Agent recenseur}
		    {Maire de Lalacelle}
		    {Lalacelle, France}
		    {}
		    {}
		\tllabelcventry{2015/1}{2015/2}{01/2015}
			{Stagiaire}
			{Renault, 4 semaines}
			{Le Mans, France}
			{Validation qualité de moyeux avant}
			{}


	\vspace*{\deletedSpace}
	\section{Langues}
		\cvdoubleitem{\bfseries Français}{langue maternelle}{\bfseries Anglais}{C1 (BULATS), usage professionnel}
		\cvitem{\bfseries Japonais}{élémentaire}

	%-------------------------------------------

	\vspace*{\deletedSpace}
	\section{Compétences informatiques}
		\cvline{Langages}{C++ 17, Java 8, C}
		\cvline{Outils}{UML, git, Make, CMake, regex}
		\cvline{Bibliothèques}{ImGui, SFML, Catch2, Boost.Asio, JavaFX}
        \cvline{Notions}{Assembleur, \LaTeX{}, PlantUML, Bash, OpenGL, Python, OpenCV}
		\cvline{IDE}{JetBrains CLion \& IntelliJ IDEA, Vim, Eclipse}

	%-------------------------------------------
	\newpage
	%-------------------------------------------

	\vspace*{\deletedSpace}
	\section{Projets}
		\subsection{Personnels}
			\cvitemcom{Bibliothèque P2P haut niveau [C++, Boost.asio]}
				{\href{https://github.com/organic-code/Breep}{>Breep}}
			\cvitemcom{Widget de Terminal pour ImGui [C++, ImGui]}
				{\href{https://github.com/Organic-Code/ImTerm/tree/master}{>ImTerm}}
			\cvitemcom{Dungeon crawler en réseau [C++, ImGui, SFML ; en cours]}
			    {\href{https://github.com/organic-code/Dungeep}{>Dungeep}}
			\cvitemcom{Chat vocal en P2P [C++, Breep]}
			    {\href{https://github.com/organic-code/Breep-distchat}{>Breep-distchat}}
			\cvitemcom{Dungeon crawler avec niveaux à génération procédurale [Java, JavaFx]}
			    {\href{https://github.com/TiWinDeTea/Raoul-the-Game}{>Raoul-the-Game}}
			\cvitemcom{Gestionaire de listes de lecture musicales YouTube [bash, youtube-dl]}
			    {\href{https://github.com/organic-code/ymph}{>ymph}}
		    \cvlistitem{Snake multijoueur en réseau [C++, SFML]}
		\subsection{Scolaire}
			\cvitemcom{Construction et rendu d'objet paramétrique semi transparent [C++, GLFW, OpenGL]}
				{\href{https://github.com/Organic-Code/OpenGL-wrapper}{>OpenGL-wrapper}}
			\cvitemcom{Adaptation du jeu de plateau Cthulhu Wars (en réseau) [Java, JavaFx, KryoNet]}
				{\href{https://github.com/TiWinDeTea/anime-warfare}{>Anime-Warfare}}
			\cvitemcom{Implémentation du radix sort (gui-console) [C]}
			    {\href{https://github.com/TiWinDeTea/RadixSort-LO27}{>RadixSort-LO27}}
			\cvitemcom{Jeu du Force 3 avec IA MinMax/AlphaBeta [C++, Qt]}
				{\href{https://github.com/Organic-Code/force3-with-bot}{>force3-with-bot}}
			\cvitemcom{Oculométrie par webcam [C++, OpenCV]}
			    {\href{https://github.com/Organic-Code/TX-eye-tracking}{>tx-eye-tracking}}
			\cvlistitem{Serveur de stockage de fichiers multi-utilisateurs (en console) [C++, sfml-network]}

	%-------------------------------------------

	\vspace*{\deletedSpace}
	\section{Centres d'intérêts}
		\cvline{L'informatique}
			{Génération procédurale, développement d'outils, bas niveau}
		\cvline{Les sciences}
			{Mathématiques, simulation de systèmes physique, physique quantique\ldots}
		\cvline{Autre}
			{Plongée, violon, randonnées}
\end{document}
